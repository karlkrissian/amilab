\documentclass{article}
\usepackage{html}
\usepackage{hyperref}
\usepackage[latin1]{inputenc}

\begin{document}

% Title Page
\title{Gaussian Filter Gui}
\author{Karl Krissian}
\thanks{
Grupo de Imagen, Tecnolog\'ia M\'edica y Televisi\'on (GIMET)\\
Universidad de Las Palmas de Gran Canaria\\
email: krissian@dis.ulpgc.es
}


\maketitle

\begin{abstract}
A simple interface for convolution with a Gaussian kernel and its derivatives.
\end{abstract}


\section{Description}
This script is an interface around the 
\htmladdnormallink{filter}
{http://serdis.dis.ulpgc.es/~krissian/HomePage/Software/AMILab/HelpNew/AMILab_scripts_doc/doc/Tokens/FILTER_rules.html}
command.
The filter command takes an input greyscale image and apply a Gaussian or Gaussian derivative convolution to it.
The convolution is applied successively along X, Y and Z directions with the corresponding given derivation order, a derivation order -1 means no processing, of order 0 means a standard Gaussian convolution, of order 1 means a convolution with the first order derivative of the Gaussian function, up to order 2.\\


\section{Parameters}

The parameters are:
\begin{itemize}
 \item input image
 \item Gaussian standard deviation
 \item Order of derivation in X
 \item Order of derivation in Y
 \item Order of derivation in Z
 \item Auto: automatic computation after changing a parameter
 \item Auto drag: automatic computation while changing a parameter with a scale
 \item Apply
\end{itemize}



\end{document}          
