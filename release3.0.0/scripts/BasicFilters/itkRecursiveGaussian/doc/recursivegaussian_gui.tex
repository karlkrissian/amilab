\documentclass{article}
\usepackage{html}
\usepackage[latin1]{inputenc}
%Párrafos
\setlength{\parskip}{1pc}
\usepackage{url}

\renewcommand{\thefootnote}{\arabic{footnote}}

\begin{document}

% Title Page
\title{ITK Recursive Gaussian Gui}
\author{ Karl Krissian \& Sara Arencibia
}
\thanks{
Grupo de Imagen, Tecnolog\'ia M\'edica y Televisi\'on (GIMET)\\
Universidad de Las Palmas de Gran Canaria\\
email: krissian@dis.ulpgc.es darkmind@gmail.com
\url{http://www.ctm.ulpgc.es/joomla/index.php}
}


\maketitle

\begin{abstract}
This script provides a user interface to do a Recursive Gaussian Filter in 2D and 3D images based on IIR convolution with an approximation of a Gaussian kernel.
\end{abstract}


\section{Quick Start}

The different steps for obtaining the Recursive Gaussian Filter are:
\begin{enumerate}
  \item Select and load the input image.
  \item Select the dimension of the image (2D or 3D).
  \item Select the sigma value, measured in world coordinates, of the Gaussian kernel. The default is 1.0.
  \item Select the flag for normalizing the gaussian over scale space.
  \item Select the Order of the Gaussian to convolve with.
  \item Run the filter.
  \item Display the result.
\end{enumerate}


\section{Equation}
Base class for computing IIR convolution with an approximation of a Gaussian kernel.
\[\frac1{\sigma\sqrt{2\pi}}\exp\left(-\frac{x^2}{2\sigma^2}\right)\]
RecursiveGaussianImageFilter is the base class for recursive filters that approximate convolution with the Gaussian kernel. This class implements the recursive filtering method proposed by R.Deriche in IEEE-PAMI Vol.12, No.1, January 1990, pp 78-87, "Fast Algorithms for Low-Level Vision"

\section{Interface tabs}

The interface is divided into several pages.

\subsection{\emph{Init} page}

\subsubsection{Input Image}
You can choose the initialization parameters.
First of all, you need to choose an input image and load it with the {\bf Load} button.
The {\bf Dimension} parameter allows to set the dimension of the image (2D or 3D).

\subsection{\emph{Param} page}
\subsubsection{Recursive Gaussian Param}
\begin{enumerate}
  \item Select the {\bf Sigma} value, measured in world coordinates, of the Gaussian kernel. The default is 1.0.
  \item Select the flag {\bf Normalize} for normalizing the gaussian over scale space. When this flag is ON the filter will be normalized in such a way that larger sigmas will not result in the image fading away.
		    \[\label{eq:GenRG}\frac1{\sqrt{2\pi}}\]
	When the flag is OFF the normalization will conserve contant the integral of the image intensity.
		    \[\frac1{\sigma\sqrt{2\pi}}\]
	For analyzing an image across Scale Space you want to enable this flag. It is disabled by default. 
  \item Select the {\bf Order} of the Gaussian to convolve with.
    \begin{enumerate}
      \item ZeroOrder is equivalent to convolving with a Gaussian. This is the default.
      \item FirstOrder is equivalent to convolving with the first derivative of a Gaussian.
      \item SecondOrder is equivalent to convolving with the second derivative of a Gaussian.
    \end{enumerate}
  \item {\bf Run} the filter.
  \item {\bf Display} the result.
  \item {\bf Save} the result.
\end{enumerate}




\subsection{\emph{Help} page}
This help.

\end{document}          
