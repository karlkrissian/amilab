\documentclass{article}
\usepackage{html}
\usepackage[latin1]{inputenc}
%Párrafos
\setlength{\parskip}{1pc}

\begin{document}

% Title Page
\title{ITK Sigmoid Filter Gui}
\author{Karl Krissian \& Sara Arencibia}
\thanks{
Grupo de Imagen, Tecnolog\'ia M\'edica y Televisi\'on (GIMET)\\
Universidad de Las Palmas de Gran Canaria\\
email: krissian@dis.ulpgc.es darkmind@gmail.com
}


\maketitle

\begin{abstract}
This script provides a user interface to compute the sigmoid function pixel-wise 
\end{abstract}


\section{Quick Start}

The different steps for obtaining the Sigmoid Filter are:
\begin{enumerate}
  \item Select and load the input image.
  \item Select the dimension of the image (2D or 3D).
  \item Select the standard deviation for the normgradient.
  \item Select the minimum and the maximum value of the output image.
  \item Select alpha and beta, constants to compute the sigmoid function.
  \item Run the filter.
  \item Display the result.
\end{enumerate}


\section{Equation}
Computes the sigmoid function pixel-wise.

A linear transformation is applied first on the argument of the sigmoid fuction. The resulting total transfrom is given by
\[ f(x) = (Max - Min)\cdot\frac1{1+e^-\frac{x-\beta}{\alpha}} \]

Every output pixel is equal to f(x). Where x is the intensity of the homologous input pixel, and alpha and beta are user-provided constants. 


\section{Interface tabs}

The interface is divided into several pages.

\subsection{\emph{Init} page}

Select and load the input image.
The {\bf Dimension} parameter allows to set the dimension of the image (2D or 3D).

\subsection{\emph{Param} page}

\begin{enumerate}
  \item Select the standard deviation for the normgradient.
  \item Select the minimum and the maximum value of the output image.
  \item Select alpha and beta, constants to compute the sigmoid function.
  \item Run the filter.
  \item Display the result.
\end{enumerate}


\subsection{\emph{Help} page}
This help.

\end{document}          
