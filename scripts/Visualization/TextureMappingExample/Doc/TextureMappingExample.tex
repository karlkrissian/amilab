\documentclass{article}
\usepackage{html}
\usepackage[latin1]{inputenc}
%Párrafos
\setlength{\parskip}{1pc}
\usepackage{url}

\renewcommand{\thefootnote}{\arabic{footnote}}

\begin{document}

% Title Page
\title{Using Texture Mapping: Example with JPEG pictures}
\author{ Karl Krissian}
\thanks{
Grupo de Imagen, Tecnolog\'ia M\'edica y Televisi\'on (GIMET)\\
Centro de Tecnolog\'ias de la Imagen (CTIM)\\
Universidad de Las Palmas de Gran Canaria\\
email: krissian@dis.ulpgc.es
\url{http://www.ctm.ulpgc.es/joomla/index.php}
}


\maketitle

\begin{abstract}
This program is an example of using Texture Mapping from VTK.
It reads all the JPEG images from the given directory and places them within a single plane
in 3D.
The user can then apply effects for a given image number:
\begin{itemize}
  \item rotate the image once on 360 degrees,
  \item show only this image in original resolution during a few seconds.
\end{itemize}
\end{abstract}




\end{document}          
