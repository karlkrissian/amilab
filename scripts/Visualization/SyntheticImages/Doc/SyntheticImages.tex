\documentclass{article}
\usepackage{/Applications/Tex/html}
\usepackage[latin1]{inputenc}


\begin{document}

% Title Page
\title{Synthetic Images Gui}
\author{Karl Krissian \& Daniel El\'ias Santana Cedr\'es}
\thanks{
Grupo de Imagen, Tecnolog\'ia M\'edica y Televisi\'on (GIMET)\\
Universidad de Las Palmas de Gran Canaria\\
email: krissian@dis.ulpgc.es
email: daniel.santana104@estudiantes.ulpgc.es
}


\maketitle

\begin{abstract}
This script provides a user interface to create synthetic images in 2D and 3D based on an analytic function (line, circle, sphere or torus).
\end{abstract}


\section{Quick Start}

For create a synthetic image follow this steps:
\begin{enumerate}
  \item Create a new input image (2D or 3D) on \emph{Main} tab.
  \item Create a new analytic function from \emph{Func} tab.
  \item Apply the method on \emph{Main} tab.
  \item You can show or save the result.
\end{enumerate}

\section{Interface tabs}

The interface is divided into several tabs.

\subsection{\emph{Main} tab}
In the \emph{Main} tab you can create a new input image indicating the x, y and z dimensions.

After to create an analytic function you can apply the method. You can choose the recursive subdivision level and two diferent methods:
\begin{itemize}
	\item Analytic Partial Surface: Compute partial effect on a 2D image using the line or circle analytic function.
	\item Analytic Partial Volume: Compute partial effect on a 3D image using the sphere or torus analytic function.
\end{itemize}

Also, you can save the image as a global variable.

\subsection{\emph{Func} tab}
In the \emph{Func} tab you can create an analytic function object. There are four functions with some parameters:
\begin{itemize}
	\item Line: The normal parameters and y axis cut point.
	\item Circle: The center and the radius.
	\item Sphere: The center and the radius.
	\item Torus: The center, the minimum radius and the maximum radius
\end{itemize}

All analytic functions have a reload option. If you check it, you can see the changes on the image when modify a parameter.

\subsection{\emph{Help} tab}
This help.


\end{document}          
