\documentclass{article}
\usepackage{html}
\usepackage[latin1]{inputenc}
%\usepackage{simplemargins}
\usepackage{fancybox}

\begin{document}

% Title Page
\centerline{\Large \bf Level Set Segmentation GUI}
 \smallkip

 \centerline{\bf Karl Krissian}
  \smallskip
  \centerline{
  \htmladdnormallink{Grupo de Imagen, Tecnolog\'ia M\'edica y Televisi\'on (GIMET)}{http://www.ctm.ulpgc.es}\\
  Univ. de Las Palmas de Gran Canaria\\
  {\it krissian@dis.ulpgc.es}
  }


%\maketitle

%{ \large Description}: \\
%\setleftmargin{1in}
%\begin{center}
% \begin{tabular}{p{9cm}}
    This script provides a user interface to segment structures in 2D and 3D images. 
    The implementation of the level set active contours is based on \cite{KrissianWestin05}.
%\end{tabular}
%\end{center}


%\setleftmargin{1in}

\section{Quick Start}

The different steps for obtaining the segmentation are:
\begin{enumerate}
  \item Select and load the input image,
  \item Select the initial level set either from spheres or from the zero-crossing of an image,
  \item Compute statistics of the intensity statistic within the initial contour,
  \item optionally tune the equation parameters,
  \item evolve the equation,
  \item display the resulting surface.
\end{enumerate}


\section{Equation}
A introduction to Level Set Equations for image processing can be found in 
\cite{Sethian99book,Sapiro01,Osher2002,OsherParagios2003}.

The level equation in a very general form is written as:
\begin{equation}\label{eq:GenLS}
%\htmlimage{scale=3}
    \left\{ \begin{array}{lcl} u(0) &=& u_0 \\%
    \frac{\partial u}{\partial t} &=& F \| \nabla u \| \end{array} \right.
\end{equation}

where $u$ is the evolving image and $F$ is the force that drives the evolution.

The force $F$ is decomposed in three terms:
\begin{equation} \label{eq:LS_terms}
    F = F_s + F_a + F_b 
\end{equation}

where:
\begin{itemize}
   \item $F_s$ is the smoothing term
   \item $F_a$ is the advection term
   \item $F_b$ is the balloon term 
\end{itemize}


\section{Interface tabs}

The interface is divided into several pages.

\subsection{\emph{Init} page}

\subsubsection{Initial Image}
You can choose the initialization parameters.
First of all, you need to choose an initial image to be segmented, and load it with the {\bf Load} button.
The {\bf Min. Intensity} parameter allows to set a minimal intensity level for the initial image, it can be useful to prevent strong contours in the background intensities that could interfere with the structures to be segmented.


\subsubsection{Initial Level Set}

\subsubsection{Intensity statistics}

\subsection{\emph{Param} page}

\subsection{\emph{Evol} page}

\subsection{\emph{Res} page}

\subsection{\emph{Help} page}
This help.

\bibliographystyle{abbrv}
\bibliography{levelset_gui}


\end{document}          
