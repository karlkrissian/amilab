\documentclass{article}
\usepackage{html}
\usepackage[latin1]{inputenc}
%Párrafos
\setlength{\parskip}{1pc}
\usepackage{url}

\renewcommand{\thefootnote}{\arabic{footnote}}

\begin{document}

% Title Page
\title{Skeletonization}
\author{ Karl Krissian
}
\thanks{
Grupo de Imagen, Tecnolog\'ia M\'edica y Televisi\'on (GIMET)\\
Centro de Tecnolog\'ias de la Imagen (CTIM)\\
Universidad de Las Palmas de Gran Canaria\\
email: krissian@dis.ulpgc.es
\url{http://www.ctm.ulpgc.es/joomla/index.php}
}


\maketitle

\begin{abstract}
Script to create skeleton from a 3D image.
Especially tested on 3D tubular structures to create the associated centerlines.
\end{abstract}


\section{Quick Start}

The different steps for obtaining the filter are:
\begin{enumerate}
  \item select the initial image
  \item select the parameters
    \begin{enumerate}
      \item Image Threshold: creates the skeleton from the pixels higher than the given threshold.
      \item Fill Holes: removes holes inside the structure before the skeletonization.
    \end{enumerate}
  \item run the filter
\end{enumerate}


\end{document}          
