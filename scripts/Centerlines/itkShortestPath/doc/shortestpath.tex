\documentclass{article}
\usepackage{html}
\usepackage[latin1]{inputenc}
%Párrafos
\setlength{\parskip}{1pc}

\begin{document}

% Title Page
\title{ITK Shortest Path Gui}
\author{Karl Krissian \& Sara Arencibia}
\thanks{
Grupo de Imagen, Tecnolog\'ia M\'edica y Televisi\'on (GIMET)\\
Universidad de Las Palmas de Gran Canaria\\
email: krissian@dis.ulpgc.es darkmind@gmail.com
}


\maketitle

\begin{abstract}
This script provides a user interface create a path in 3D.
\end{abstract}


\section{Quick Start}

The different steps for obtaining the Path are:
\begin{enumerate}
  \item Select and load the input image.
  \item Select and load the speed image if you want to use it.
  \item Select the start point and the end point.
  \item If you want to save the fastmarching output mark Save FastMarching Output.
  \item Select epsilon value.
  \item Select maxcost value.
  \item If you load the speed image mark Use Speed, then select the minimum and maximum intensity and the input speed value.
  \item Select the step size.
  \item Select the maxlength.
  \item Run the filter.
  \item Display the result.
\end{enumerate}


\section{Detailed Description}
Creates a path starting at a point and following the displacements given by the input vector field image. 
If the input is scalar the corresponding vector field is computed from its gradient by local linear interpolation.
The tracking stops in one of the following cases:
\begin{enumerate}
  \item The maximal distance is reached.
  \item The closest voxel to the current location has a negative intensity.
  \item The current gradient is very low.
\end{enumerate}


\section{Interface tabs}

The interface is divided into several pages.

\subsection{\emph{Init} page}

You can choose the initialization parameters.
First of all, you need to choose an input image and load it with the {\bf Load} button. 
Then, you need to choose a speed image and load it with the {\bf Load} button.

\subsubsection{Extremities}
Select the start point with the {\bf Set Start Point} button. An image appear, select with the central mouse button the seed point and then press the same button once more.
Select the end point with the {\bf Set End Point} button. An image appear, select with the central mouse button the seed point and then press the same button once more.


\subsection{\emph{Param} page}
If you want to show or save the FastMarching output image, you must mark ``Save FastMarching Output''. Now, you can use the image.
Select the epsilon value. This is use in the FastMarching solution and for create the path.
Select the maxcost, maximal time.

\subsubsection{Input Intensity-based Speed}
If you want to use speed image, you must mark ``Use speed''.
Then, select the minimum and the maximum intensity that are used.
Select the input speed value.


Select the step size for the evolution.
Select the maxlength, maximal Euclidean distance of the path.

Run with the {\bf Run} button.
Display the result with the {\bf Display} button.

\subsection{\emph{Help} page}
This help.

\end{document}          
