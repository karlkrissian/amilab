\documentclass{article}
\usepackage{html}
\usepackage[latin1]{inputenc}
%Párrafos
\setlength{\parskip}{1pc}

\begin{document}

% Title Page
\title{ITK Vesselness Gui}
\author{Karl Krissian \& Sara Arencibia}
\thanks{
Grupo de Imagen, Tecnolog\'ia M\'edica y Televisi\'on (GIMET)\\
Universidad de Las Palmas de Gran Canaria\\
email: krissian@dis.ulpgc.es darkmind@gmail.com
}


\maketitle

\begin{abstract}
This script provides a user interface to provide a vesselness measure for tubular objects from the hessian matrix
\end{abstract}


\section{Quick Start}

The different steps for obtaining the Vesselness are:
\begin{enumerate}
  \item Select and load the input image.
  \item Select the dimension of the image (2D or 3D).
  \item Select the sigmaMin and the sigmaMax value.
  \item Select the number of scales.
  \item Run the filter.
  \item Display the result.
\end{enumerate}


\section{Detailed Description}
Line filter to provide a vesselness measure for tubular objects from the hessian matrix. The filter takes as input an image of hessian pixels (SymmetricSecondRankTensor pixels) and preserves pixels that have eigen values $ \lambda_3 $ close to 0 and $\lambda_2$ and $\lambda_1$ as large negative values. (for bright tubular structures).

\[ \lambda_1 < \lambda_2 < \lambda_3 \]

The filter takes into account that the eigen values play a crucial role in discrimintaitng shape and orientation of structures.

    * Bright tubular structures will have low $\lambda_1$ and large negative values of $\lambda_2$ and $\lambda_3$.
    * Conversely dark tubular structures will have a low value of $\lambda_1$ and large positive values of $\lambda_2$ and $\lambda_3$.
    * Bright plate like structures have low values of $\lambda_1$ and $\lambda_2$ and large negative values of $\lambda_3$
    * Dark plate like structures have low values of $\lambda_1$ and $\lambda_2$ and large positive values of $\lambda_3$
    * Bright spherical (blob) like structures have all three eigen values as large negative numbers
    * Dark spherical (blob) like structures have all three eigen values as large positive numbers

This filter is used to discriminate the Bright tubular structures.

A filter to enhance structures using Hessian eigensystem-based measures in a multiscale framework.

The filter evaluates a Hessian-based enhancement measure, such as vesselness or objectness, at different scale levels. The Hessian-based measure is computed from the Hessian image at each scale level and the best response is selected.

Minimum and maximum sigma value can be set using SetMinSigma and SetMaxSigma methods respectively. The number of scale levels is set using SetNumberOfSigmaSteps method. Exponentially distributed scale levels are computed within the bound set by the minimum and maximum sigma values

The filter computes a second output image (accessed by the GetScalesOutput method) containing the scales at which each pixel gave the best reponse.

A filter to enhance M-dimensional objects in N-dimensional images.

The objectness measure is a generalization of Frangi's vesselness measure, which is based on the analysis of the the Hessian eigen system. The filter can enhance blob-like structures (M=0), vessel-like structures (M=1), 2D plate-like structures (M=2), hyper-plate-like structures (M=3) in N-dimensional images, with M<N. The filter takes an image of a Hessian pixels ( SymmetricSecondRankTensor pixels pixels ) and produces an enhanced image. The Hessian input image can be produced using itkHessianSmoothedRecursiveGaussianImageFilter.

\section{Interface tabs}

The interface is divided into several pages.

\subsection{\emph{Init} page}

\subsubsection{Input Image}
You can choose the initialization parameters.
First of all, you need to choose an input image and load it with the {\bf Load} button.

The {\bf Dimension} parameter allows to set the dimension of the image (2D or 3D).

\subsection{\emph{Param} page}
\subsubsection{Vesselness Param}
Select the minimum and the maximum of sigma value.
Select the number of scales levels.
Now run the filter with the {\bf Run} button.
You can see the result with the {\bf Display} button.
You can save the result with the {\bf Save} button.

\subsection{\emph{Help} page}
This help.

\begin{thebibliography}{1}
\bibitem{VS} {\em "3D Multi-scale line filter for segmentation and visualization of curvilinear structures in medical images"}, Yoshinobu Sato, Shin Nakajima, Hideki Atsumi, Thomas Koller, Guido Gerig, Shigeyuki Yoshida, Ron Kikinis.
\bibitem{VS2} Frangi, AF, Niessen, WJ, Vincken, KL, & Viergever, MA (1998). {\em Multiscale Vessel Enhancement Filtering}. In Wells, WM, Colchester, A, & Delp, S, Editors, MICCAI '98 Medical Image Computing and Computer-Assisted Intervention, Lecture Notes in Computer Science, pages 130-137, Springer Verlag, 1998.
\end{thebibliography}

\end{document}          
