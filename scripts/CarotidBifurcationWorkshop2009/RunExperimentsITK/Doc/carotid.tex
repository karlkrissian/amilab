\documentclass{article}
\usepackage{html}
\usepackage[latin1]{inputenc}
%Párrafos
\setlength{\parskip}{1pc}
%Paquete para hipervínculos
\usepackage{hyperref}
%Configuración del coloreado de los hipervínculos
\hypersetup{
  colorlinks,
  citecolor=black,
  filecolor=black,
  linkcolor=black,
  urlcolor=blue %Las url se muestran en azul
}

\begin{document}

% Title Page
\title{ITK CarotidBifurcationWorkshop2009 Gui}
\author{Karl Krissian \& Sara Arencibia}
\thanks{
Grupo de Imagen, Tecnolog\'ia M\'edica y Televisi\'on (GIMET)\\
Universidad de Las Palmas de Gran Canaria\\
email: krissian@dis.ulpgc.es darkmind@gmail.com
}


\maketitle

\begin{abstract}
This script provides a user interface to compute Carotid Bifurcation Workshop solution ussing ITK libraries.
\end{abstract}

\section{Detailed Description}
For more references go to \cite{IJ}.

\section{Interface tabs}
For use this script you need to check the configuration in carotidchallenge\_config.amil. You must to specify:
\begin{enumerate}
  \item Data\_dir = data path folder
  \item Groundtruth\_dir = groundtruth path folder (lumen)
  \item Results\_dir = results path folfer
  \item Evaluation\_bindir = evaluation build path folder
  \item Evaluation\_scriptdir = evaluation scripts path folder
\end{enumerate}

The interface is divided into several pages.

\subsection{\emph{Input} page}
\subsubsection{data information}
If you want to use the advanced mode you need to mark ``Advanced Mode''. Now you can change wherever you want on the Adv tab, modify parameters, try differents cases...


When you mark ``Advanced Mode'' some tabs are activated:
\begin{enumerate}
  \item Crop, use to crop the input image. The image is cropped by default, but you can change the limits.
  \item Dir, you can change the different path folder and if you want to run the process without confirm any info dialog you need to mark No\_interaction.
  \item Adv, see the explanation below.
\end{enumerate}

Select the Datacenter: Erasmus MC, Hadassah or Louis Pradel.


Select the Datatype: Training, Testing or On-site. Note that testing mode not have groundtruth data.


Select the Data Number.


To read and crop the data use the {\bf Read Data} button.


To read the groundtruth use the {\bf Read GT} button.

\subsection{\emph{Prob+Speed} page}
\subsubsection{LocalStats}
Use the {\bf Apply} button to compute the LocalStats.
\subsubsection{Vesselness}
\subsubsection{Vessel intensity range}
You can change the Min and Max intensity range to compute Vesselness. Min = 1150 and Max = 1600 are set by default.

Use the {\bf Vesselness} button to compute the Vesselness solution.


You can save the result using {\bf Save} button.


Then you can read the output vesselness solution using the {\bf Read} button.
\subsection{\emph{Run} page}
\subsubsection{Paths(Vesselness+Prob)}
\subsubsection{Path}
Use {\bf Create} button to create the paths, external and internal carotid.


Use {\bf Save} button to save the results.


Use {\bf Read} button to read the results from disk.
\subsubsection{Junction}
You can change the parameter Threshold and Resample but there are set by default to work correctly for this Workshop.


Select {\bf Junction} button to compute the junction of the carotids.


You can see the result using the {\bf Display} button. An image is showed, the green sphere is the junction.


Save the result using {\bf Save} button.
\subsubsection{LevelSets from paths}
Select {\bf External} button to compute a Level Set solution for the external carotid path.


Select {\bf Internal} button to compute a Level Set solution for the internal carotid path.


To display the result you need to save it before.

\subsection{\emph{Adv} page}
To know how to use each tab see the help of each tab.

\subsection{\emph{Help} page}
This help.

\begin{thebibliography}{1}
\bibitem{CLS} \href{http://cls2009.bigr.nl/index.php}{Workshop Rules}
\bibitem{IJ} \href{http://www.insight-journal.org/browse/publication/672}{Insight Journal}
\end{thebibliography}

\end{document}

